\documentclass[11pt,a4paper]{article}

% Packages
\usepackage{geometry}
\usepackage{setspace}
\usepackage{graphicx}
\usepackage{lipsum}
\usepackage{url}
\usepackage{hyperref}
\usepackage{csquotes}
\usepackage{xcolor}


% Page formatting
\geometry{a4paper, margin=1in}
\renewcommand{\baselinestretch}{1.0}
\setlength{\parskip}{1em}
\pagenumbering{arabic}

% Bibliography
\usepackage[
    backend=biber,
    style=nature
]{biblatex}
\addbibresource{references.bib}

\begin{document}

% Title
\title{AI-assisted Design of virus-binding proteins for the International Genetically Engineered Machine comptetion}
\author{Tony MAKDISSY\thanks{Learning Planet Institute} \and 
Amir PANDI\thanks{Affiliation and role} \and 
Ariel LINDNER\thanks{Affiliation and role} \and
Ernest MORDRET\thanks{Affiliation and role} \and
Helena SHOMAR\thanks{Affiliation and role}}




\date{August 2023 -- November 2023}
\maketitle

% Abstract
\begin{abstract}
    \begin{displayquote}
        Contains in clear langage the problem statement, the indication of methodology, the main findings and the principal conclusion.
    \end{displayquote}  
\end{abstract}

% General Context
\section*{General Context}
\begin{displayquote}
    Describe what your laboratory, company, institution is doing in general. 
Describe precisely what your team does and what their expertise is in a few lines. (don’t copy -paste a boiler).
Who are the key people you worked with and how did they acquire their expertise (mention their degrees, professional experience…). Mention any other people you or your team collaborated with during your internship (limited to the particular topic you worked on). 
Add 3-4 lines on your integration to the institution and the work environment.
Connection to the sustainable development goals (list the SDGs), if any.
\end{displayquote}

% what is iGEM
This report details my involvement in the Paris-Bettencourt team's 
collaborative participation in the International Genetically Engineered 
Machine (iGEM) contest \cite{igem_main}. The iGEM competition, held annually, 
is a globally recognized Synthetic Biology competition, uniting 
participants across three distinct age groups: high school, undergraduate, 
and graduate students \cite{igem_description}.

% villages
The topics covered by the competition are divided into 15 themes called villages \cite{igem_villages}.
Paris-Bettencourt project, named "Lubritect", was part of the Therapeutics village.

% Lubritect project
Lubritect is an innovative solution that combines mucin-based hydrogel 
with AI-generated protein structures, aiming to reduce the transmission 
of sexually transmitted infections (STIs). This approach leverages de novo 
protein design for versatility against various pathogens. Lubritect was designed
as an answer for alarming statistics regarding STIs, with high incidence 
(1 million new sexually transmitted infections every day), 
prevalence (80\% of sexually active individuals will acquire human papillomavirus by 45)
and disease burden (82,000 deaths in 2019 from hepatitis B) \cite{paris_bettencourt_project}.

% Team composition
Paris-Bettencourt team is hosted by Learning Planet Institute,
consisting from 7 Learning Planet students, and 3 Non-Learning Planet students.
Each having a specific role like wetlab, drylab, or human practices \ldots etc. 
% My role
My main role in this project was generating and \emph{in-silico} testing of new protein structures to bind to the targets of interest.
% Supervisors
The team is supervised by 4 supervisors: Ariel Lindner, Ernest Mordret, Helena 
Shomar, and Amir Pandi \cite{paris_bettencourt_team}.
\textcolor{red}{Should I talk more about the supervisors ?}



% Introduction
\section{Introduction}
\begin{displayquote}
    Past research or work in the field.
    Define precisely what are the questions, objectives and tasks you were given. Connect them with a scientific or technical context.
    What are the approaches generally used to solve the problem ? (Reference them)
    What are the underlying assumptions or hypotheses, if any?
    Question raising.
    What are their limitations ? 
    Is there a gap ? Which one ?
    Purpose of the present research or work and experimental strategy chosen to address your scientific question
    Literature review.
    It is an echo to the points raised in the introduction. You can reference findings and describe the state-of-the art.
\end{displayquote}


% Methods
\section{Methods}
\begin{displayquote}
    (1-2 pages with figures if relevant): (Methods is a section valid for any internship, not only research oriented, experimental or theoretical. You did something according to a method, which is what you should understand and present here.)
Present in detail the tools, techniques and methods you have used and why (giving a list of library’s or equipment’s name is not a presentation). 
Describe the scientific and/ or technical background with clear explanations, references, equations and/ or schematics or pictures. 
\end{displayquote}

% Results
\section{Results}
\begin{displayquote}
    (2-3 pages with figures):
Precisely describe the results of your work with clear explanations of the analysis, schematics and figures.
Describe what worked, what didn’t and why.
\end{displayquote}


% Discussion
\section{Discussion}

\begin{displayquote}
    (1 page): 
Review findings.
Discuss outcomes.
Do your results make sense ? 
Evaluate them
Do they provide elements towards solving your problem ? Which ones ?
Do they open up new questions (scientific or technical).

\end{displayquote}



% Conclusion
\section{Conclusion}

\begin{displayquote}
    (½ to 1 page maximum):
    Conclude regarding the missions and tasks you were given and the results you obtained . Mention if they will be used by your team. 
    What are the limitations of your results ?
    What are the future directions (questions, implementation…) ? 
\end{displayquote}    

% Acknowledgments


\section*{others}

\textbf{Check the Internship validation: Guidelines 2023-2024}

\printbibliography

\end{document}